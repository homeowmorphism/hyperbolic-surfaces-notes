\textbf{Goal} 

Prove the Nielson Realization Problem.\marginnote{Kerckhoff (paper in annals)}

\textbf{Sources}
\begin{itemize}
\item{Katok - Fuchsian groups.}
\item{Farb and Margalit - A primer on mapping class groups}
\item{Hubbard - Teichm\"uller Theory Vol 1}
\end{itemize}

\textbf{Course webpage}

math.toronto.edu/mbourque

\section{Chapter 0 : Overview}

\subsection{Classification of close oriented surfaces}
Surfaces have symmetries and we can try to classify them. 

Any finite subgroup of $\IsomP(S^2)$\marginnote{$\IsomP$ is the set of orientation preserving isometries, $S^2$ is the 2-dimensional sphere.} isormorphic to $\ints_n$, $D_n$, \\ $\Rot(\text{tetrahedron})$, $\Rot(\text{cube})$ (which is equal to $\Rot(\text{octahedron})$ or $\Rot(\text{dodecatedron})$.

What about the symmetry of the torus?

We can have $\ints_4$ symmetries or $\ints_6$ symmetries. That's it!

What about hyperbolic surfaces?

Q1: What is the largest number of orientation preserving isometries of a closed hyperbolic surface of genus $g$?

$[\Isom(S): \IsomP(S)] = 2$. 

A1: by a theorem of Hurwitz, $|\IsomP(S)| \leq 42 \underbrace{|\chi(S)|}_{|2 - 2g|}$

The bound is attained by infinitely many $g$ but not all. 

Q2: What is the largest possible order of such an isometry?

A2: $4g + 2$ - this is realized by taking a regular 4g + 2 - g on $\hplane$ with opposite sides glued by isometries.

Q3: Which groups can arise as $\IsomP(S)$?

A3: Any finite group.

\begin{definition}
	The maping class group of a surface S 
	$h$
